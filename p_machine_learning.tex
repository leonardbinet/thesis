\chapter{Machine learning challenges}
\label{cha:results}




\section{Natural Language Processing}
\subsection{Word2Vec models}

When dealing with natural language, one of the main challenges is to  
\subsubsection*{Continuous Bag of Words}
\subsubsection*{Skip-Gram}

\subsection{Fasttext}

\subsubsection{Model}
Suppose a previous layer $h$ (taking a vector x of given size in ouput space of size d) as follow:
\begin{align}
	h 
	&= 
	\begin{bmatrix} 
		h_1 \\
		h_2 \\
		\vdots \\
		h_{\textit{dim}}
	\end{bmatrix}\\
	&= A^{\top}x
\end{align}


\begin{align}
	h_j = A^{(j)\top}x
\end{align}

With $A$ being the matrix of weights from input to hidden layer, of size $(\textit{dim}, V)$, ie dimension of vector embeddings $\times$ vocabulary size.
\begin{align}
	A &= \left[\begin{array}{cccc}| & | & | & | \\ A^{(1)} & A^{(2)} & \cdots & A^{(\textit{dim})} \\ | & | & | & | \end{array}\right]\\
	A &= \left[
		\begin{array}{cccc} 
		  	-- & A_{(1)} & -- \\
		  	-- & A_{(2)} & -- \\
		  	&\vdots 	\\
		  	-- & A_{(V)} & --
		\end{array}\right]
\end{align}



Let's denote:

\begin{align}
	s_k  = B^{(k)\top} h = \sum_{j=1}^{\textit{dim}} B^{(k)\top}_j h_j 
\end{align}

we can express $f$ as, $\forall k \in [1, K]$:

\begin{align}
	f_k  = \sigma(s_k) = \sigma( \sum_{j=1}^{d} B^{(k)\top}_j h_j) 
\end{align}

Lets consider the following error on one sample $(x, y)$:

\begin{align}
	E = - \sum_{k=1}^K
  			  	\left\{
				    \begin{array}{ll}
				        \log (f_k) & \mbox{if } y_k =1 \\
				        \log (1 - f_k) & \mbox{if } y_k =0
				    \end{array}
				\right.
\end{align}

This error is relevant since the minimization of this error is equivalent to maximizing the maximum of (log-)likelihood.
Plus, it is interpretable: if the $k^{th}$ component $y_k$ is:
\begin{itemize}
	\item positive ($y_k=1$): then the loss increase if $f_k$ is near 0, and decrease if $f_k$ is near 1.
	\item negative ($y_k=0$): then the loss decrease if $f_k$ is near 0, and increase if $f_k$ is near 1.
\end{itemize}



We can represent it as the following multilayer network:

\begin{tikzpicture}[shorten >=1pt,->,draw=black!50, node distance=\layersep]
    \tikzstyle{every pin edge}=[<-,shorten <=1pt]
    \tikzstyle{neuron}=[circle,fill=black!25,minimum size=17pt,inner sep=0pt]
    \tikzstyle{input neuron}=[neuron, fill=yellow!50];
    \tikzstyle{output neuron}=[neuron, fill=blue!50];
    \tikzstyle{hidden neuron}=[neuron, fill=red!50];
    \tikzstyle{true neuron}=[neuron, fill=green!50];
    \tikzstyle{annot} = [text width=4em, text centered]

    % Draw the input layer nodes
    \foreach \name / \y in {1,...,7}
    % This is the same as writing \foreach \name / \y in {1/1,2/2,3/3,4/4}
        \node[input neuron, pin=left:$x_{\y}$ ] (I-\name) at (0,-\y) {};

    % Draw the hidden layer nodes
    \foreach \name / \y in {1,...,3}
        %\path[yshift=0.5cm] node[hidden neuron] (H-\name) at (\layersep,-\y cm) {};
        \node[hidden neuron] (H-\name) at (\layersep,-1 -\y) {};

    % Draw the output layer node
    % \node[output neuron,pin={[pin edge={->}]right:Output}, right of=H-3] (O) {};

	\foreach \name / \y in {1,...,4}
        \node[output neuron, pin=right:$f_{\y}$ ] (O-\name) at (\layersep*2,-0.5 -\y) {};

    % Draw the true layer node

	\foreach \name / \y in {1,...,4}
        \node[true neuron, pin=right:$y_{\y}$ ] (T-\name) at (\layersep*3,-0.5 -\y) {};


    % Connect every node in the input layer with every node in the
    % hidden layer.
    \foreach \source in {1,...,7}
        \foreach \dest in {1,...,3}
            \path (I-\source) edge (H-\dest);

    % Connect every node in the hidden layer with the output layer
    \foreach \source in {1,...,3}
        \foreach \dest in {1,...,4}
	        \path (H-\source) edge (O-\dest);

    % Annotate the layers
    \node[annot,above of=H-1, node distance=2.5cm] (hl) {Hidden layer: $h$ of size \textit{dim}};
    \node[annot,left of=hl] {Input layer: $x$ (size $V$)};
    \node[annot,right of=hl] (ol) {Output layer: $f$ of size $K$};
    \node[annot,right of=ol] {True layer: $y$ of size $K$};

    \node[annot] (A) at (\layersep/2,-6) {$A$};
    \node[annot] (B) at (\layersep*3/2,-5) {$B$};

\end{tikzpicture}

\subsubsection*{Gradient retropropagation}


\begin{align}
	\frac{ \partial E } { \partial f_k } = 
		\left\{
		    \begin{array}{ll}
		        - \frac{1}{f_k} & \mbox{if } y_k =1 \\
		        \frac{1}{1 - f_k} & \mbox{if } y_k =0
		    \end{array}
		\right.
\end{align}


\begin{align}
	\frac{ \partial E } { \partial s_k } 
		=  
		\frac{ \partial E } { \partial f_k } \cdot \frac{ \partial f_k } { \partial s_k } 
		&=
		\left\{
		    \begin{array}{ll}
		        - \frac{1}{f_k} \cdot f_k (1 - f_k)& \mbox{if } y_k =1 \\
		        \frac{1}{1 - f_k} \cdot f_k (1 - f_k)& \mbox{if } y_k =0
		    \end{array}
		\right. \\
		&=
		\left\{
		    \begin{array}{ll}
		       f_k - 1 & \mbox{if } y_k =1 \\
		       f_k & \mbox{if } y_k =0
		    \end{array}
		\right. \\
		&= f_k - y_k
\end{align}



\begin{align}
	\frac{\partial E}{\partial B_i^{(k)}} 
	= 
	\frac{\partial E}{\partial s_k} \cdot \frac{\partial s_k}{\partial B_i^{(k)}} 
	= 
	h_i (f_k - y_k)
\end{align}


Trick: derivate $E$ in regard to $s_k$:
\begin{align}
	\frac{\partial E}{\partial h_j} 
	&= 
	\sum_{k=1}^K \frac{\partial E}{\partial s_k} \cdot \frac{\partial s_k}{\partial h_j} \\
	&= 
	\sum_{k=1}^K B_j^{(k)} (f_k - y_k)
\end{align}


\begin{align}
	\frac{\partial E}{\partial A_i^{(j)}} 
	&= 
	\frac{\partial E}{\partial h_j} \cdot \frac{\partial h_j}{\partial A_i^{(j)}} \\
	&= 
	x_i \sum_{k=1}^K B_j^{(k)} (f_k - y_k)
\end{align}


\subsubsection*{Gradient descent}

At each sample $(x, y)$, given a learning rate $\mu$, we update weight as follow:

$B$ weights:
\begin{align}
	B_i^{(k)\mbox{new}} \leftarrow B_i^{(k)\mbox{old}} - \mu (h_i (f_k - y_k))
\end{align}

$A$ weights:
\begin{align}
	A_i^{(j)\mbox{new}} \leftarrow A_i^{(j)\mbox{old}} - 
	\mu 
	\left(
		x_i \sum_{k=1}^K B_j^{(k)\mbox{new}} (f_k - y_k) 
	\right)
\end{align}

\subsection{Fasttext parameters implementation}

List of all parameters:
\begin{itemize}
	\item \textit{epoch}
	\item \textit{lr}
	\item \textit{lrUpdateRate}
	\item \textit{dim}
	\item \textit{ws}
	\item \textit{loss}
	\item \textit{neg}
	\item \textit{minCountLabel}
	\item \textit{minCount}
	\item \textit{minn}
	\item \textit{maxn}
	\item \textit{bucket}
	\item \textit{t}
\end{itemize}

\subsection*{Dictionnary}

\subsection*{Learning rate}

Fasttext implementation lets you choose two parameters to control $\mu$ over time:
\begin{itemize}
	\item \textit{lr}: sets $\mu$ at initialization
	\item \textit{lrUpdateRate}: how continuous \textit{vs} per steps \textit{lr} decays
\end{itemize}

The learning rate $\mu$ decrease linearly, from initial given parameter $lr$ to zero.





\pagebreak
\section{Multilabel classification}
\subsection{Review of major algorithms}
\subsubsection{Problem Transformation Approaches}
\subsubsection{Adaptative Approaches}
\subsubsection{Neural network}

\subsection{Multiclass/multilabel scoring metrics}
\subsubsection{Regular binary metrics}


Evaluation metrics should take into account the class imbalance between churners and non-churners. A greater cost should be associated with false negatives than with false positives: misidentify a potential churner costs far more to a company than identifying a non-churner as churner. Moreover, we are more interested in perfectly identifying those customers who are most likely to churn than perfectly identifying \textit{all} the potential churners.

\subsubsection{Binary classification metrics}

As we are in a binary classification problem, several metrics can be used to assess the performance of the models.

Lets define some naming/metrics used in the following:

{\ttfamily
\begin{table}[H]
    \centering
    \begin{tabular}{ll}
        \toprule
        $TP / FP$          &    $true/false\ positives$ \\
        $TN / FN$       &    $true/false\ negatives$  \\
        $P$       &    $TP+FN$ \\
        $N$    &    $TN + FP$ \\
        $recall$      &    ${TP} / {P}$ \\
        $precision$        &   ${TP} / {(TP+FP)}$ \\
        \bottomrule
    \end{tabular}
\end{table}
}

Precision, recall and accuracy are often used to measure the classification quality of binary classifiers.

In customer churn prediction, the misclassification of a churner may result in the loss of the customer but the misclassification of a non-churner may result in some extra marketing cost.  Because the former is more costly than the latter, recall is a more important measure than precision \cite{VdP05}.
While precision and recall measure how accurate the methods can identify the observations in a single class, the AUC measures how well the methods discriminate the two classes.

Precision measures that fraction of examples classified as churner that are truly churner.

Recall measures the fraction of well classified churners.


The problem of churn differs from classical classification/prediction setups, notably in the objectives to optimize on. Although having a good recall, for instance, is important (we want as many churners well classified as possible), one should really take into account the \textit{business application}: a typical use of a churn prediction model is to target potential future churners and to launch a marketing campaign to retain them. In such cases, we are willing to tolerate greater overall error, in return for better identifying the most likely future churners for further attention.

Contacting a large proportion of the customers is expensive: one wants to maximize the number of potential churners reached while minimizing the total number of contacted customers. Metrics such as \textit{lift} (\ref{sub:lift}) express very well this business application constraint.


\textbf{F-score}

The $F_1$ is a measure of a classifier accuracy, computed as the harmonic mean of precision and recall: $$F_1= 2. \frac{precision.recall}{precision+recall}$$\tabularnewline
This definition assign the same weight to precision and recall but in our case, recall is more valuable than precision: we want all the churners to be correctly classified and type I error has less monetary impact (people may be targeted for a marketing campaign even if they weren't about to churn).

To handle these constraints, the $F_\beta-score$ is adapted: in consists in the \textit{weighted} harmonic mean of recall and precision, assigning $\beta$ times as much importance to recall as precision: $$F_\beta = (1+\beta^2) \frac{precision.recall}{\beta^2.precision+recall}$$

\textbf{Calibration plot} 

When dealing with probabilistic classifiers, one of the signs that a suitable classification model has been found is also that predicted probabilities (scores) are well calibrated, that is that a fraction of about $p$ of events with predicted probability $p$ actually occurs. Calibration plot is a method that shows us how well the classifier is calibrated \cite{VC06}: $$x = true\ probability,\ y=predicted\ probability$$

True probabilities are calculated for (sub)sets of examples with the same predicted score $P(y=1|X) \in [c-\delta, c+\delta]$: $$ p_{true}^c = \frac{P_{sub}^c}{P_{sub}^c + N_{sub}^c}$$ with $P_{sub}^c, N_{sub}^c$ being the proportion on positives and negative examples in a given subset with predicted probabilities $[c-\delta, c+\delta]$. The calibration plot of a perfectly calibrated classifier will be a diagonal.



\subsubsection{Adaptation to multiclass-multilabel: global / local metrics}



\section{Calibration}

\subsection{Predictions precision/recall estimation}

\section{Invalid/sparse data}

\subsection{Strategies to handle sparse data}
Consider recall rather than precision. The presence of a lot a FN will impact precision, not recall.

\subsection{Iterative approach to bootstrap new classes}
Manually label some products with new classes, then focus validation on these products. The validated suggestions will be used in next classification workflow.

After some iterations, there might be enough occurences to ensure sufficient scores.


\section{Misc}

\subsection{Dive into kind tree-structure}
\subsection{Enforce better calibration}
\subsection{Tuning strategies}
